\documentclass[a4paper,11pt,titlepage]{jarticle}
\usepackage[dvipdfmx]{graphicx}
\usepackage{listings}
\usepackage{amsmath}
\usepackage{fancybox,ascmac}
\usepackage{url}

\title{知能情報実験II : 第七回レポート}
\author{175751C 宮城孝明}
\date{\today}

\begin{document}
\maketitle
\tableofcontents
\clearpage
\section{実験目的}
  アセンブラプログラムの作成,ハンドアセンブル(アセンブラプログラムを
人手で機械語プログラムに直すこと),実行の書く作業を実際に行うことにより,
アセンブラプログラミングの流れを習得する.また,コンパイラとアセンブラの
違いや,高水準言語とアセンブラ言語の違いについて理解することを目的とする.

\section{実験概要}
  この実験では,KUE-CHIP2を用いてアセンブラプログラミングを人の手で機械語
プログラムに変換し,その結果をオシロスコープで確認した.初めのプログラム結果
では,オシロスコープの波形はノコギリ型を示す形で表示された.そして,課題の波形
では矩形波や山形波,菱形波の3種類であった.これらのプログラムを組み立てるために,
まずは頭の中のイメージを直接図にして描いてみたりなどした.そうすることで,
頭の中を整理していく.イメージが出来てきたら,プログラムを実際に書いていく.
出来上がったプログラムを直接機械語に翻訳していく.この1つ1つの工程をこなすことで,
作成者自身のプログラミング力が向上し,さらには目的の高水準言語とアセンブラ言語との
違いをより明確的に把握することができる.

\section{実験結果}
 \begin{enumerate}
  \item 実験(1)について
  \par
  本実験の結果に関しては,報告を省略する.
  \item 実験(2),(3)の結果について
  \par
  \begin{figure}[htbp]
\centering
\includegraphics[width=50mm]{3.png}
\label{矩形波の波形}\\
\caption{矩形波の波形}
  \end{figure}
  \clearpage
  \lstinputlisting[numbers=left,breaklines=true,basicstyle=\ttfamily\footnotesize,
  frame=sing,caption=矩形波の波形,label=1]{sample1.s}
\par
      それでは,矩形波の波形プログラムを見ていこう.初めに,ACCとACCのXORを使用してACCの値を0にする.
    次に,データ領域(40H)に格納していたFFの値を変数IXに代入する.そして,OUTを用いてACCの値を表示
    していく.ACCの値は0のため表示では下に棒線が続くようになる.次の処理で変数IXの値を1ずつ減らしていく.
    そして,IXの値が0になるまで,繰り返し処理を続けていく.IXの値が0になれば次の処理が開始される.
    ACCとIXはに,IXに代入したデータ領域(40H)をまた格納する.そして,ACCの値を表示する.ACCの値は,FFのため
    表示では上に棒線が続くようになる.そして,IXを1ずつ減らしていく.そして,IXの値が0になるまで処理を
    続ける.IXが0になればまた,アドレス10に戻り再度繰り返し処理を開始していく.

  \begin{figure}[htbp]
  \centering
  \includegraphics[width=50mm]{4.png}
  \label{山形波の波形}\\
  \caption{山形波の波形}
  \end{figure}
  \clearpage
  \lstinputlisting[numbers=left,breaklines=true,basicstyle=\ttfamily\footnotesize,
  frame=sing,caption=山形波の波形,label=2]{sample2.s}
\par
      それでは,山形波の波形プログラムを見ていこう.初めに,データ領域(40H)の値FFを変数IXに格納する.
    そして,変数ACCにもデータ領域(40H)を格納する.そして,ACCの値を表示する.そして,ACCの値を1ずつ減らし
    ACCの値が0になるまで,表示と減少を繰り返していく.これにより,ノコギリ波とは逆向きの波形を作ることができる.
    ACCが0になり次の処理を行う.初めはACCの値を表示し,0になったACCに1ずつ増やす処理を行う.そして,
    初めに値を格納したIXを1ずつ減らしていく.そして,IXが0になるまで繰り返す.これにより,ノコギリ波と同じ
    波形を作ることができる.よって,全体図としては,山形波ができる.そして,また初めの処理に戻り,繰り返していく.
 \end{enumerate}


\section{考察}
 \begin{enumerate}
  \item 実験(1),(2),(3)の考察について
  \par
      今回の方法では,2つ以上の同時出力が可能でない.菱形型波形を作成時に何度か同時出力を試したが,
    出来なかった.菱形は,上の点と下の点を条件分岐する前に行わないといけない.しかし,そのような方法
    を取ってしまうと,次の処理に影響が出てしまう.よって,今回課題として出ていた菱形波形の完成図には
    微細のズレがある.これにより,同時出力をしないといけない処理がある波形(例えば,円や二重線,二次関数)
    は出力出来ない可能性がある.それ以外の波形は出力することは可能かもしれない.
  \item その他の考察について
  \par
        アセンブラ言語のOUTが出力する変数はACCだけである.そのため,変数1つだけでどうにか出力を
      やりくりした.変数が限られたプログラムはしたことがなく,とても苦労した.いつも使用している言語
      では多くの変数が利用でき,どれだけ楽にプログラムができていたのかがよくわかった.この実験を
      通して,普段のプログラム作成時には,変数をもっと意識して作っていくことが必要であると認識を改める
      ようにしないといけない.
  \par
  \end{enumerate}
\section{調査課題}
\begin{enumerate}
  \item パイプライン処理について
  \par
  \begin{figure}[htbp]
  \centering
  \includegraphics[width=100mm]{sample1.png}
  \label{通常の逐次制御方式}\\
  \caption{通常の逐次制御方式}
  \end{figure}
      まず,図3を見て欲しい.Aは命令の取り出し,Bは命令の解読,Cは対象データ読み出し,Dは命令の実行で
    分けられたステージを3回処理している.通常の逐次処理は,1つの命令が終わするまで次の命令を実行
    できない.このような流れ作業のため,非効率に回っていることがよくわかる.\par
  そのため,全体の処理を効率に行うために使われる方法としてパイプライン処理がある.\par
  図4を見てみる.
  \begin{figure}[htbp]
  \centering
  \includegraphics[width=100mm]{sample2.png}
  \label{パイプライン処理}\\
  \caption{パイプライン処理}
  \end{figure}
\par
      この処理は,各ステージが終了したあとに,次の命令が始まるようになっている.例えば,1番目の命令の
    ステージAの処理が始まり,終了したら2番目の命令のステージAも始まる.1番目の命令の ステージBが終了
    したら3番目の命令のステージAの処理が始まる.このように処理していくことで,通常の逐次処理よりも
    早く処理を終わらせることができる.例えば,ステージ1つの処理に1クロックかかるとしたら,通常の
    逐次制御方式には12クロック,パイプライン処理は6クロックと半分の時間で処理する.\par
      この処理の欠点としては,命令を先読みして行なっているため,条件分岐の命令が出てきた場合
    先読みした部分が無駄になってしまうことがある.これを分岐ハザードと呼ぶ.

  \item
    アセンブラ,コンパイラ,インタプリタについて\par
      言語プロセッサは,ソースプログラムをパソコンが理解できる機械語に変換してくれるプログラム
    である.\par
      アセンブラは,機械語と1対1に対応している言語である.初期の頃のコンピュータによく使用されていた.
    現在でも,ハードウェアを制御するマイクロコンピュータの世界で活躍している.うまく利用できれば
    コンピュータの性能をフル活用ができ,メモリ容量を最小に抑えることもできる.小規模のシステムで
    利用されている.\par
  代表的な言語としては,アセンブラ言語である.\par
      コンパイラは,人間が書いたソースコードを一括で実行可能な形式に変換する言語である.一度で実行
    可能な形式に変換するため,実行速度はインタプリタ言語よりも速い.コンパイラによっては,実行効率が
    悪いものになる.メモリ容量は,アセンブラよりも多くなる.開発効率が高いため,大規模な開発に向いている.\par
      代表的な言語としては,C言語,Java言語などが取り上げられる.\par
      インタプリタは,ソースプログラムを機械語に変換せずそのまま,1行ずつ翻訳し実行を行う.そのため,
    実行終了にはコンパイラ言語と比べ,とても時間がかかる.しかし,インタプリタ言語はプログラムが全て
    完成していなくても実行可能であり,デバックが容易である.\\
      代表的な言語としては,python,PHPなどが上げられる.\\
      高水準言語は,人間の言葉に近い言語で人間が理解しやすい.メモリや入出力の制御を行わなくてよい.\par
      低水準言語は,コンピュータが理解しやすくて,メモリや入出力装置などの周辺機器の制御ができるため,
    高レベルな処理が可能である.
\end{enumerate}

\section{感想}
      今回は,実験の課題3つが厳しくすぐに終わらせることができず,さらに追加課題も終わらせることができなかった.
    自分の考えすぎる点も問題だと感じた.プログラミングは,試行錯誤が大事であり,失敗しても何度も繰り返し,
    成功できるまでやり続けることが大切だと改めて認識した.
\section{参考文献}
 \begin{thebibliography}{99}
 \bibitem{kihonjouhou}基本情報技術者; きたみりゅうじ,キタミ式 イラストIT塾 平成30年度 基本情報処理技術者試験
基本情報技術者 2018
  \bibitem{processor}processor; http://www.it\-license.com/Program\_language/Language\_processor.html
  \bibitem{picfun}picfun; http://www.picfun.com/prog04.html
  \bibitem{wcnixwu6yec}wcnixwu6yec; https://wcnixwu6yec.wiki.fc2.com/wiki/高水準言語と低水準言語
 \end{thebibliography}

\end{document}
